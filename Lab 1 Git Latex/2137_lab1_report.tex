% Digital Logic Lab 1 Report
% Created: 2020-01-16, Ashlie Lackey

%==========================================================
%=========== Document Setup  ==============================

% Formatting defined by class file
\documentclass[11pt]{article}

% ---- Document formatting ----
\usepackage[margin=1in]{geometry}	% Narrower margins
\usepackage{booktabs}				% Nice formatting of tables
\usepackage{graphicx}				% Ability to include graphics

%\setlength\parindent{0pt}	% Do not indent first line of paragraphs 
\usepackage[parfill]{parskip}		% Line space b/w paragraphs
%	parfill option prevents last line of pgrph from being fully justified

% Parskip package adds too much space around titles, fix with this
\RequirePackage{titlesec}
\titlespacing\section{0pt}{8pt plus 4pt minus 2pt}{3pt plus 2pt minus 2pt}
\titlespacing\subsection{0pt}{4pt plus 4pt minus 2pt}{-2pt plus 2pt minus 2pt}
\titlespacing\subsubsection{0pt}{2pt plus 4pt minus 2pt}{-6pt plus 2pt minus 2pt}

% ---- Hyperlinks ----
\usepackage[colorlinks=true,urlcolor=blue]{hyperref}	% For URL's. Automatically links internal references.

% ---- Code listings ----
\usepackage{listings} 					% Nice code layout and inclusion
\usepackage[usenames,dvipsnames]{xcolor}	% Colors (needs to be defined before using colors)

% Define custom colors for listings
\definecolor{listinggray}{gray}{0.98}		% Listings background color
\definecolor{rulegray}{gray}{0.7}			% Listings rule/frame color

% Style for Verilog
\lstdefinestyle{Verilog}{
	language=Verilog,					% Verilog
	backgroundcolor=\color{listinggray},	% light gray background
	rulecolor=\color{blue}, 			% blue frame lines
	frame=tb,							% lines above & below
	linewidth=\columnwidth, 			% set line width
	basicstyle=\small\ttfamily,	% basic font style that is used for the code	
	breaklines=true, 					% allow breaking across columns/pages
	tabsize=3,							% set tab size
	commentstyle=\color{gray},	% comments in italic 
	stringstyle=\upshape,				% strings are printed in normal font
	showspaces=false,					% don't underscore spaces
}

% How to use: \Verilog[listing_options]{file}
\newcommand{\Verilog}[2][]{%
	\lstinputlisting[style=Verilog,#1]{#2}
}




%======================================================
%=========== Body  ====================================
\begin{document}

\title{ELC 2137 Lab 1: Git and LaTeX Intro}
\author{Ashlie Lackey}

\maketitle


\section*{Summary}

This lab requires a student to follow an introduction to Git and LaTeX in order to submit a report. In doing this, the student should be able to describe the basic Git process, synchronize local and remote files in a repo using GitHub, and create a lab report in LaTeX that includes images, code, lists, tables, and section headings.   


\section*{Q\&A}

\begin{enumerate}
\item What is your GitHub user name?
	
	My user name is Alackey21.
\item  What LaTeX environment produces a bulleted (non-numbered) list?
	
	The bulleted list without numbers is an itemize environment.
\item  Write the equation $y(t) = 1/2 e^t$ using LaTeX equation formatting.
	
	$y(t) = \frac{1}{2} e^t$
\item What is the shortcut key for compiling your LaTeX document?

	
	To only compile and not view your LaTeX document simply press the F6 button. Use F5 to compile and view the document. 
\end{enumerate}

\section*{Results}

This figures corresponding to the results section are included on the following page.
\begin{figure}[ht]\centering
	\begin{tabular}{c|c|c}
		\toprule
		Binary & Hex & Decimal \\
		\midrule
		0000 & 0 & 0 \\
		0010 & 2 & 2 \\
		0100 & 4 & 4 \\
		0110 & 6 & 6 \\
		1000 & 8 & 8 \\
		1010 & A & 10 \\
		\bottomrule
	\end{tabular} 
	
	\includegraphics[scale=0.75, trim=18.5cm 15.5cm 0.5cm 4.5cm, clip]{lab1_example_screenshot}	
	\caption{ This Table displays the simulation waveform and table for reproduction.}
	\includegraphics[scale=0.60]{Screenshot}	
	\caption{ This screenshot corresponds to step 1.5.3 in the Lab 1 instructions.}
\end{figure}

\clearpage
\section*{Code}


\begin{lstlisting}[style=Verilog,
caption=Direct Verilog code example,
label=code:ex 
]
module example
#(parameter BITS=4)
(
input [BITS-1:0] in0, in1,
input sel,
output [BITS-1:0] out
);

// Choose in1 or in0
out = sel ? in1: in0; 
endmodule
\end{lstlisting}

\end{document}
